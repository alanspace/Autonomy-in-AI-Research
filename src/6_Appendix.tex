\section{Appendix}
\subsection{Brief History of Artificial Intelligence and Robotics}
The history of artificial intelligence (AI) and robotics is a fascinating journey marked by key milestones that have transformed theoretical concepts into practical applications integral to modern society. From early philosophical inquiries to the development of sophisticated humanoid robots, this paper summarizes the evolution of AI and robotics, highlighting the shift from rule-based systems to data-driven models and the integration of AI with robotics. The following sections outline major developments, supported by authoritative references.

\subsubsection{Early Foundations (1940s--1950s)}
The roots of AI and robotics trace back to the 1940s, when foundational ideas about intelligent machines began to emerge. In 1942, Isaac Asimov introduced the ``Three Laws of Robotics'' in his short story \textit{Runaround}, providing an ethical framework for intelligent machines \citep{Asimov1942}. In 1943, Warren S. McCulloch and Walter H. Pitts published a seminal paper, ``A Logical Calculus of the Ideas Immanent in Nervous Activity,'' introducing artificial neural networks, which became a cornerstone of AI research \citep{McCullochPitts1943}. Norbert Wiener's 1948 book, \textit{Cybernetics: Or Control and Communication in the Animal and the Machine}, explored feedback mechanisms, influencing AI and control systems \citep{Wiener1948}. In 1950, Alan M. Turing published ``Computing Machinery and Intelligence,'' proposing the Turing Test to evaluate machine intelligence, a concept that remains central to AI philosophy \citep{Turing1950}.

\subsubsection{Birth of AI as a Field (1950s--1960s)}
The formal establishment of AI as a discipline occurred in 1955 with the Dartmouth Summer Research Project on Artificial Intelligence, proposed by John McCarthy, Marvin Minsky, Nathaniel Rochester, and Claude Shannon \citep{McCarthy1955}. This conference is widely regarded as the birth of AI, setting ambitious goals for machine intelligence. In 1958, McCarthy developed LISP, a programming language tailored for AI research, which remains in use today \citep{McCarthy1960}. In 1959, Arthur L. Samuel coined the term ``machine learning'' while developing a checkers-playing program, marking the inception of this subfield \citep{Samuel1959}. Between 1964 and 1966, Joseph Weizenbaum created ELIZA, a natural language processing program that simulated conversation, attempting to pass the Turing Test \citep{Weizenbaum1966}. In robotics, the development of Shakey the Robot (1966--1972) by the Stanford Research Institute marked a significant milestone, as it was one of the first mobile robots capable of reasoning about its environment using AI techniques \citep{Nilsson1984}.

\subsubsection{AI Winter and the Rise of Machine Learning (1970s--1990s)}
The optimism of early AI research was tempered by challenges in the 1970s, notably the first ``AI Winter'' starting in 1973. James Lighthill's report, ``Artificial Intelligence: A General Survey,'' criticized AI's progress, leading to reduced funding in the United States and United Kingdom \citep{Lighthill1973}. During this period, AI research shifted toward more practical applications, such as expert systems, which gained prominence in the 1980s. The development of the backpropagation algorithm for training neural networks in 1986 by David E. Rumelhart and colleagues revitalized interest in neural networks, laying the groundwork for modern machine learning \citep{Rumelhart1986}.

\subsubsection{Deep Learning and Modern AI (2000s--Present)}
The 21st century has witnessed a renaissance in AI, driven by the advent of deep learning. In 2012, AlexNet's victory in the ImageNet competition demonstrated the power of convolutional neural networks, sparking the deep learning revolution \citep{Krizhevsky2012}. In 2015, Google's AlphaGo defeated the world champion in the complex game of Go, showcasing advanced AI capabilities through deep neural networks and reinforcement learning \citep{Silver2016}. The development of large language models (LLMs), such as OpenAI's GPT series and Meta's Llama, has transformed natural language processing, enabling machines to generate coherent and contextually aware text \citep{RussellNorvig2009}. In robotics, the integration of AI has led to the creation of sophisticated humanoid robots, such as Boston Dynamics' Atlas, known for its agility, and Tesla's Optimus, designed for general-purpose tasks. These robots exemplify the convergence of AI and robotics, enabling machines to perform complex physical tasks in dynamic environments \citep{RussellNorvig2009}.