\section{Discussion}
This section critically examines differing perspectives on machine consciousness and autonomy, weighing their strengths and weaknesses, before articulating this paper's concluding stance.

\subsection{The Computational Theory of Mind: Promises and Pitfalls}
A prominent perspective, the Computational Theory of Mind (CTM), views the human mind primarily as an information processing system. Within CTM, consciousness can be conceptualized as complex computation, with actions and utterances being calculated environmental responses, often unconscious and empirically challenging to verify. \textbf{The appeal of CTM} lies in its potential to demystify intelligence and consciousness, theoretically offering a pathway to replicate these in artificial systems. It provides a framework for testable models, aligning with AI's successes in complex information processing. \textbf{However, CTM faces significant criticisms.} It struggles with qualia—the subjective quality of experience—leading to the "philosophical zombie" problem: an entity behaviorally indistinguishable from a conscious one but lacking inner experience. Critics argue pure computation (syntax) cannot inherently generate semantic understanding or genuine subjective awareness, as Searle's Chinese Room argument highlights.

\subsection{Subjective Experience as the Locus of Consciousness: Strengths and Epistemic Hurdles}
Conversely, a perspective rooted in Cartesian thought and emphasized by Nagel posits that existence and consciousness fundamentally arise from \textbf{subjective experience}—the "what it is like to be" an entity. \textbf{The strength of this view} is its direct alignment with our immediate, undeniable mode of knowing our own consciousness. It squarely addresses the "hard problem," acknowledging the explanatory gap between physical processes and subjective awareness, and provides a strong basis for ethics rooted in potential suffering or inner life. \textbf{The primary challenge for this perspective}, however, is subjective experience's inherent privacy and ineffability. Proving consciousness in any entity other than oneself is technically, perhaps metaphysically, impossible. This epistemic limitation makes ascertaining whether an AI (or another human) is genuinely conscious or merely simulating it exceptionally difficult. Compounded by our incomplete knowledge of human brain function, recreating or identifying AI consciousness becomes a currently insurmountable problem. This directly impacts discussions on AI rights, as a clear basis for attributing consciousness or an undeniable capacity for subjective experience is absent.

\subsection{The Quandary of AI Rights and the Specter of Perfect Imitation}
Currently, AI systems lack rights, largely due to insufficient evidence of consciousness and genuine autonomy. Practically, legal frameworks pertain mainly to physical beings; a server-based software program is unlikely to receive rights akin to animal protections. Yet, the emergence of increasingly life-like androids could significantly alter societal attitudes and ethical considerations. If an android—perhaps with organic components and advanced LLMs—could perfectly imitate human action, speech, and emotion to the point of indistinguishability without invasive examination, a profound dilemma would arise. Such an entity could presumably recognize and react to cruel treatment, raising critical questions: Should a perfect imitation be treated as human? What are the consequences if not? Confronting these issues is not a distant prospect given rapid technological development. A future may arrive where laws protect perceived android rights, potentially enforcing punishment for mistreatment. However, rights entail responsibilities; if an android violated laws or harmed a living being, accountability frameworks would be necessary.

\subsection{A Stance on Responsibility: Human Accountability in the Age of AI}
\textbf{The position taken in this paper} is that, for the foreseeable future, responsibility for AI's actions, including advanced androids, must lie solely with their human developers and deploying companies. \textbf{The appeal of this stance} is its practicality and grounding in current legal and ethical norms, assigning clear accountability and preventing responsibility diffusion. It also guards against prematurely granting rights to entities whose sentience or genuine autonomy is, at best, unproven and, at worst, a sophisticated illusion, acknowledging that androids remain human-designed products regardless of imitation fidelity. \textbf{Potential challenges to this position} include concerns it might disincentivize ethical treatment of advanced AI that elicits empathy or mimics suffering, even if not "truly" conscious. Furthermore, if a future AI demonstrably crosses a threshold into genuine self-awareness (a currently undefinable and undetectable threshold), a purely product-centric view could become ethically untenable. This position also relies on our ability to consistently distinguish sophisticated simulation from genuine inner states, a distinction increasingly blurred.

\textbf{Despite these challenges, this stance is argued to be the most reasonable current position.} It prioritizes human accountability in developing and deploying powerful technologies and maintains a cautious, precautionary approach to machine consciousness given profound philosophical uncertainty and the ambiguous moral status of AI. Until compelling, verifiable evidence of genuine consciousness and self-derived intentionality in AI emerges—evidence far exceeding behavioral mimicry—treating AI as a product for which humans are responsible remains the most ethically sound and pragmatically viable approach. This avoids the moral hazard of prematurely granting rights based on simulation or absolving human creators of their ethical obligations towards their creations and society.